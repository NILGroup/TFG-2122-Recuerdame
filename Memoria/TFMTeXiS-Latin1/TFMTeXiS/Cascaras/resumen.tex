% !TeX encoding = ISO-8859-1
% +--------------------------------------------------------------------+
% | Copyright Page
% +--------------------------------------------------------------------+

\chapter*{Resumen}

El alzheimer es una enfermedad mental degenerativa que afecta a millones de personas en todo el mundo. A pesar de ello no existe ning�n remedio y medicina que cure esta enfermedad mental. El �nico remedio que tenemos para la ralentizaci�n del avance de la enfermedad son las terapias de reminiscencia. Estas terapias consisten en pensar o hablar de actividades, eventos y experiencias pasadas del paciente mediante recuerdos tangibles de su ni�ez o su adolescencia que sirvan de ``puerta'' a recuerdos m�s recientes. Todos estos recuerdos e historias de una persona son lo que se llaman "Historias de vida". La idea de este proyecto es crear una aplicaci�n que  facilite a los terapeutas la realizaci�n de estas terapias, haci�ndolas mas �giles y r�pidas. Este documento contiene la informaci�n recogida, estudios y procesos realizados para la creaci�n de una aplicaci�n que ayudar� a realizar dichas terapias.  Para ello, consultaremos a distintos expertos que nos ayudar�n a enfocar la aplicaci�n a sus necesidades.

En este repositorio de GitHub se encuentran todos los ficheros fuente tanto de la aplicaci�n, como los ficheros de la memoria en latex:

https://github.com/NILGroup/TFG-2122-Recuerdame

\section*{Palabras clave}
   
\noindent Alzheimer, Reminiscencia, Vida, Memoria, Aplicaci�n.

   


