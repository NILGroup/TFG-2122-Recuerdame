% !TeX encoding = ISO-8859-1
% +--------------------------------------------------------------------+
% | Copyright Page
% +--------------------------------------------------------------------+

\chapter*{Resumen}

Este proyecto presenta la creaci�n de una aplicaci�n que facilite a los terapeutas la realizaci�n de terapias basadas en reminiscencia para tratar a pacientes con alzheimer, haci�ndolas mas �giles y r�pidas. 

Para la realizaci�n de la aplicaci�n hemos usado un dise�o centrado en el usuario, ya que durante su desarrollo hemos contado con expertos que nos han ido guiando en sus expectativas y requisitos. Para obtener los requisitos de la aplicaci�n nos reunimos con unos terapeutas expertos en el tema y nos encargamos de hacerles preguntas que nos permitieran saber que es lo que necesitaban en la aplicaci�n. Despu�s realizamos un an�lisis de estos datos para conseguir los requisitos. Tras conseguir todos los requisitos, cada uno de los autores de este proyecto, realiz� un prototipo de dise�o de la aplicaci�n. Para realizar un dise�o competitivo, de esta fase obtuvimos una primera versi�n del dise�o de la aplicaci�n que fue evaluada por uno de los expertos.

La aplicaci�n creada es una aplicaci�n web responsive, con una estructura Modelo Vista Controlador. La que hemos creado mediante lenguajes como HTML, CSS, PHP, JavaScript y conectada a una Base de datos no relacional.

Finalmente nos reunimos con los expertos para realizar la evaluaci�n final de aplicaci�n. Los resultados de esta evaluaci�n fueron bastantes positivos ya que, aunque a la aplicaci�n le queda muchos detalles que retocar, los expertos se mostraron bastante contentos del resultado final.


\section*{Palabras clave}
   
\noindent Alzheimer, Reminiscencia, Historias de vida, Dise�o centrado en el usuario.

   


