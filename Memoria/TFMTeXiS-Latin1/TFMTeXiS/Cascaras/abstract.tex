% !TeX encoding = ISO-8859-1

\chapter*{Abstract}

This project presents the creation of an application that facilitates therapists to perform reminiscence-based therapies to treat Alzheimer's patients, making them faster and more agile.  

For the development of the application we have used a user-centered design, working with end users who have been providing us with their expectations and requirements throughout the whole development process. To obtain the requirements of the application we met with two therapists and asked them questions that allowed us to know what they needed in the application. Then, We performed an analysis of this data to get the requirements. 
After getting all the requirements, each of the authors of this project, made a prototype design of the application, to perform a competitive design iteration. The result of this iteration was evaluated by a end user of the application.

The application created is a responsive web application, with a Model View Controller structure created using languages such as HTML, CSS, PHP, JavaScript.

Finally we met with the end users to perform the final evaluation of the application. The results of this evaluation were quite positive because, although the application still has some details to be retouched, the experts were quite happy with the final result. The result of the SUS questionnaire filled out by the therapists gave our application a score of 85, which means that the application has an acceptable usability.


\section*{Keywords}

\noindent Alzheimer's, Reminiscence, Life stories, User-centered design.



