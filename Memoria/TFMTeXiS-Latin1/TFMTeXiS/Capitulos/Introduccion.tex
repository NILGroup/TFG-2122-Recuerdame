% !TeX encoding = ISO-8859-1

\chapter{Introducci�n}
\label{cap:introduccion}

\chapterquote{Frase c�lebre dicha por alguien inteligente}{Autor}

\section{Motivaci�n}
Introducci�n al tema del TFM.

\subsection{Explicaciones adicionales}
Si quieres cambiar el \textbf{estilo del t�tulo} de los cap�tulos, abre el fichero \verb|TeXiS\TeXiS_pream.tex| y comenta la l�nea \verb|\usepackage[Lenny]{fncychap}| para dejar el estilo b�sico de \LaTeX.

Si no te gusta que no haya \textbf{espacios entre p�rrafos} y quieres dejar un peque�o espacio en blanco, no metas saltos de l�nea (\verb|\\|) al final de los p�rrafos. En su lugar, busca el comando  \verb|\setlength{\parskip}{0.2ex}| en \verb|TeXiS\TeXiS_pream.tex| y aumenta el valor de $0.2ex$ a, por ejemplo, $1ex$.

El siguiente texto se genera con el comando \verb|\lipsum[2-20]| que viene a continuaci�n en el fichero .tex. El �nico prop�sito es mostrar el aspecto de las p�ginas usando esta plantilla. Quita este comando y, si quieres, comenta o elimina el paquete \textit{lipsum} al final de \verb|TeXiS\TeXiS_pream.tex|

\subsubsection{Texto de prueba}


\lipsum[2-20]