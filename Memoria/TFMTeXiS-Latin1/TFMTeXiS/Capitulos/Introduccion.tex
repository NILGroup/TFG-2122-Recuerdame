% !TeX encoding = ISO-8859-1

\chapter{Introducci�n}
\label{cap:introduccion}

\chapterquote{La demencia se come el pensamiento del enfermo y a su vez destroza los sentimientos de los que lo quieren y lo cuidan}{Dr. Nolasc Acar�n Tusell}

En este cap�tulo del documento se presentar�n los motivos y objetivos que se encuentran detr�s de este TFG. Introduciendo de manera breve el alz�imer y como afecta en nuestro d�a a d�a.



\section{Motivaci�n}

El problema de la p�rdida de memoria afecta a un amplio abanico de la poblaci�n, desde personas con deterioro cognitivo leve hasta los casos m�s graves de demencias como el Alzheimer. La p�rdida de memoria constituye un problema significativo para el bienestar de los pacientes, pero tambi�n un importante obst�culo para los familiares y cuidadores que se ocupan de �l. Los enfoques no farmacol�gicos del problema en el �mbito de la terapia ocupacional, basados en ayudar a los pacientes a revisar su propia historia personal o Historia de Vida, han mostrado resultados positivos, tanto como medio para mantener frescos los recuerdos como de ayuda para ejercitar la cognici�n b�sica, que se sabe que retrasa el deterioro de la memoria. Estos enfoques implican que la persona con demencia haga un registro personal de las experiencias, personas y lugares m�s importantes de su vida o que se anime a la persona con demencia a hablar sobre un periodo, evento o tema de su pasado. Se ha demostrado que ambas cosas ayudan a mantener la autoestima, la confianza y el sentido de s� mismo de las personas, as� como a mejorar las interacciones sociales con los dem�s.

\section{Objetivos}
El objetivo principal de este TFG es crear una aplicaci�n que muestre a los pacientes, familiares y/o terapeutas la Historia de Vida que un paciente ha registrado. La aplicaci�n servir� para ayudar tanto al paciente a revisar su propia historia personal y as� mantener frescos sus recuerdos como a los terapeutas a preparar terapias basadas en reminiscencia. Para conseguirlo se proponen los siguientes objetivos m�s espec�ficos:

\begin{itemize}
	\item Se har�n reuniones con expertos y usuarios finales para dise�ar una aplicaci�n que cubra las necesidades reales de los usuarios finales y tenga en cuenta sus limitaciones.
	\item Se seguir� una metodolog�a de desarrollo �gil que permita ir dando valor al producto poco a poco y tener en todo momento un working product.
	\item Se realizar�n evaluaciones con los usuarios finales para medir el impacto de lo desarrollado y as� poder perseverar en las buenas decisiones y cambiar de estrategia respecto a las cuestiones que no funcionen.
\end{itemize}