% !TeX encoding = ISO-8859-1

\chapter{Introducci�n}
\label{cap:introduccion}

\chapterquote{La demencia se come el pensamiento del enfermo y a su vez destroza los sentimientos de los que lo quieren y lo cuidan}{Dr. Nolasc Acar�n Tusell}

Cuando en la vida cotidiana hablamos de la demencia o el Alzheimer no llegamos a darnos cuenta de la magnitud con la que afecta en el d�a a d�a estas enfermedades. Afectan muchos m�s de lo que pensamos, de hecho, en el mundo existen aproximadamente 55 millones de casos y en Espa�a ya son 800.000 casos confirmados. Mayormente afecta a personas mayores de 65 a�os. Si sumamos esto al progresivo envejecimiento de la poblaci�n y al aumento de la esperanza de vida, se da lugar a que cada d�a sean m�s casos registrados. \citep{article_2}

Al no existir una cura total, los m�dicos aplican en sus pacientes distintas terapias que ayudan al retraso de los efectos degenerativos que producen esta enfermedad. Entre estas terapias se encuentra la terapia de reminiscencia.

Esta terapia consiste en ejercitar la memoria del paciente intentado rememorar momentos importantes de su pasado, siendo ayudados mediante fotos, v�deos o audios de estos recuerdos. Numerosos estudios han demostrado que este tipo de terapias ayudan a frenar el desarrollo de la enfermedad.



\section{Motivaci�n}
Se estima que m�s de 50 millones de personas en el mundo padecen demencia, unos 10 millones nuevos casos cada a�o. Donde 2 de cada 3 est�n causados por el Alzheimer. Una enfermedad neurol�gica que aumenta exponencialmente a partir de los 65 a�os de edad, y que en Espa�a afecta a m�s de 800000 personas.

A futuro si se mantuviese constante este crecimiento se podr�a llegar a m�s de 130 millones de casos en el mundo.

Al ser un problema sanitario de primer orden, existen diferentes terapias con el fin de poder ralentizar o evitar la aparici�n temprana de esta enfermedad. Y de aportar bienestar a los pacientes que la padecen. Algunas que podemos enumerar son: terapia ocupacional, musicoterapia, estimulaci�n cognitiva, etc.

Y cabe destacar entre estas las terapias por reminiscencia. 

La reminiscencia es una t�cnica que favorece la evocaci�n de recuerdos y sucesos del pasado conect�ndolos en el tiempo y el presente, con el fin de mejorar el deterioro cognitivo que supone el envejecimiento. 

Por esta raz�n, surge la necesidad y principal objetivo de este TFG, de contribuir en este tipo de terapias con una aplicaci�n inform�tica que facilite su aplicaci�n por parte de los terapeutas. Y que ayude al bienestar de pacientes y familiares.

Esta aplicaci�n permitir� que los pacientes puedan registrar su propia historia de vida y puedan volver a visualizarla en cualquier momento. 

De esta forma, toda la informaci�n de cada paciente estar� organizada y estructurada de tal forma que la terapia es mucho m�s sencilla tanto para el paciente como para los terapeutas y familiares.

\section{Objetivos}