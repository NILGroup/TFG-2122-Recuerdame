% !TeX encoding = ISO-8859-1

\chapter{Introducci�n}
\label{cap:introduccion}

\chapterquote{La demencia se come el pensamiento del enfermo y a su vez destroza los sentimientos de los que lo quieren y lo cuidan}{Dr. Nolasc Acar�n Tusell}

Cuando en la vida cotidiana hablamos de la demencia o el Alzheimer no llegamos a darnos cuenta de la magnitud con la que afecta en el d�a a d�a estas enfermedades. Afectan muchos m�s de lo que pensamos, de hecho, en el mundo existen aproximadamente 55 millones de casos y en Espa�a ya son 800.000 casos confirmados. Mayormente afecta a personas mayores de 65 a�os. Si sumamos esto al progresivo envejecimiento de la poblaci�n y al aumento de la esperanza de vida, se da lugar a que cada d�a sean m�s casos registrados. \citep{article_2}

Al no existir una cura total, los m�dicos aplican en sus pacientes distintas terapias que ayudan al retraso de los efectos degenerativos que producen esta enfermedad. Entre estas terapias se encuentra la terapia de reminiscencia.

Esta terapia consiste en ejercitar la memoria del paciente intentado rememorar momentos importantes de su pasado, siendo ayudados mediante fotos, v�deos o audios de estos recuerdos. Numerosos estudios han demostrado que este tipo de terapias ayudan a frenar el desarrollo de la enfermedad.



\section{Motivaci�n}
El Alzheimer es una enfermedad que afecta a un gran n�mero de personas en todo el mundo y se ha comprobado que las terapias de reminiscencia son muy �tiles y ayudan a mejorar la calidad de vida de las personas que sufren esta enfermedad. Por ello, es importante crear una aplicaci�n donde los pacientes puedan registrar su propia Historia de Vida y donde puedan volver a visualizarla en cualquier momento. De esta forma, toda la informaci�n de cada paciente est� organizada y estructurada de tal forma que la terapia es mucho m�s sencilla tanto para el paciente como para los terapeutas y familiares.

\section{Objetivos}