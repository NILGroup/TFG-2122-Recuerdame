% !TeX encoding = ISO-8859-1

\chapter{Introduction}
\label{cap:introduction}

\chapterquote{Dementia eats away at the patient's thoughts and in turn destroys the feelings of those who love and care for him or her}{Dr. Nolasc Acarín Tusell}

This chapter will present the reasons behind this TFG as well as the objectives we set at the beginning of its development. In addition, the methodology followed during the whole TFG and the structure of the report will be presented.



\section{Motivation}
The problem of memory loss affects a wide range of the population, from people with mild cognitive impairment to the most severe cases of dementia such as Alzheimer's disease. There are approximately 55 million cases of dementia worldwide, where 2 out of 3 are caused by Alzheimer's disease. In Spain there are already 800,000 confirmed cases. This problem largely affects people over 65 years of age, which, added to the progressive aging of the population and the increase in life expectancy, results in more and more cases being registered every day. \citep{article_2}. Not only should we be concerned about the above data, but also, in the future, if this growth remains constant, it could reach more than 130 million cases in the world. Memory loss is a significant problem for the well-being of patients, but also a major obstacle for family members and caregivers who look after them.

Since there is no total cure, physicians apply various therapies to their patients that help delay the degenerative effects of the disease. Non-pharmacological approaches to the problem in the field of occupational therapy, based on helping patients to review their own personal history or Life History, have shown positive results, both as a means of keeping memories fresh and as an aid in exercising basic cognition, which is known to slow memory decline. These approaches involve the person with dementia making a personal record of the most important experiences, people and places in his or her life or being encouraged to talk about a period, event or topic from his or her past. Both have been proven to help maintain people's self-esteem, confidence and sense of self, as well as improve social interactions with others.

\section{Objectives}
The main objective of this TFG is to create an application that shows patients, family members and/or therapists the Life History that a patient has recorded. The application will serve to help both the patient to review their own personal history and thus keep their memories alive and the therapists to prepare reminiscence-based therapies. In order to achieve this, the following more specific objectives are proposed:

\begin{itemize}
	\item Meetings will be held with experts and end users to design an application that meets the real needs of end users and takes into account their limitations.
	\item Evaluations will be carried out with the end users to measure the impact of what has been developed and thus be able to persevere with good decisions and change strategy with respect to those issues that do not work.
\end{itemize}

The development of this TFG will allow us to know the complete process of realization of a project from the definition of functional requirements with users to the implementation of the final solution through various technologies. This will allow us to acquire the necessary skills for its development, both functional and technical, thus completing the skills acquired during the degree.


\section{Methodology}
For the development of the TFG, periodic meetings will be held with the project directors through Google Meet to see the progress of the project and define the tasks for the next meeting.

Team members will have meetings via Discord and will use the Trello\footnote{https://trello.com/b/7kQ1WZN8/tfg.} tool to organize the tasks of each team member with a Kanban board. Each task can be in 3 statuses: task list, in progress or done. In this way, the progress and status of the tasks can be checked and the whole team will know the status of the project at all times.

The application will be carried out under a user-centered design in order to satisfy all the needs of the end users of the application. For this purpose, the following will be carried out:
\begin{enumerate}
	\item Meetings will be held with experts in the field to design the application based on the real needs of the end users.
	\item A design of the application will be carried out in consultation with the end users.
	\item The application will be evaluated with the end user.
\end{enumerate}

\section{Structure of the Report}

Chapter 2 of this report presents the State of the Art. This chapter will review all the concepts related to the project to be carried out such as: Alzheimer's disease, Alzheimer's stages, Reisberg global impairment scale, reminiscence-based therapies, life histories and related works. In order to present each section, a previous research has been carried out, which has allowed us to have a wide knowledge about the subject of the project.
		
Chapter 3 explains everything related to the capture of requirements: How the expert interviews were scheduled, conducted and developed and the requirements obtained from them.

Chapter 4 presents everything related to the design of the application. The prototype of each of the authors, the competitive design and its evaluation with experts.

Chapter 5 discusses how the application was made.

Chapter 6 presents the therapist evaluation of the final application and the results obtained.

Finally, Chapter 7 provides conclusions and future work.









