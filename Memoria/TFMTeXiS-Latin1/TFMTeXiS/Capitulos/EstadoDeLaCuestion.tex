% !TeX encoding = ISO-8859-1

\chapter{Estado de la Cuesti�n}
\label{cap:estadoDeLaCuestion}

En el estado de la cuesti�n es donde aparecen gran parte de las referencias bibliogr�ficas del trabajo. Una de las formas m�s c�modas de gestionar la bibliograf�a en {\LaTeX} es utilizando \textbf{bibtex}. Las entradas bibliogr�ficas deben estar en un fichero con extensi�n \textit{.bib} (con esta plantilla se proporcionan 3, dos de los cuales est�n vac�os). Cada entrada bibliogr�fica tiene una clave que permite referenciarla desde cualquier parte del texto con los siguiente comandos:

\begin{itemize}
\item Referencia bibliografica con cite: \cite{ldesc2e}
\item Referencia bibliogr�fica con citep: \citep{notsoshort}
\item Referencia bibliogr�fica con citet: \citet{latexAPrimer}
\end{itemize}

Es posible citar m�s de una fuente, como por ejemplo \citep{latexCompanion,LaTeXLamport,texKnuth}

Despu�s, latex se ocupa de rellenar la secci�n de bibliograf�a con las entradas \textbf{que hayan sido citadas} (es decir, no con todas las entradas que hay en el .bib, sino s�lo con aquellas que se hayan citado en alguna parte del texto).

Bibtex es un programa separado de latex, pdflatex o cualquier otra cosa que se use para compilar los .tex, de manera que para que se rellene correctamente la secci�n de bibliograf�a es necesario compilar primero el trabajo (a veces es necesario compilarlo dos veces), compilar despu�s con bibtex, y volver a compilar otra vez el trabajo (de nuevo, puede ser necesario compilarlo dos veces). 
