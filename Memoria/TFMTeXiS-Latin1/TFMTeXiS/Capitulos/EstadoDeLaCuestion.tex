% !TeX encoding = ISO-8859-1

\chapter{Estado de la Cuesti�n}
\label{cap:estadoDeLaCuestion}

A lo largo de esta secci�n se proceder� a realizar una explicaci�n de lo que se trata la Enfermedad del Alzheimer y su clasificaci�n, as� como el uso de terapias de reminiscencia y las Historias de Vida.

\section{Enfermedad del Alzheimer}

Se denomina Alzheimer a la demencia gradual que causa p�rdidas de memoria y habilidades cognitivas impidiendo la realizaci�n de la vida cotidiana con normalidad. El Alzheimer es el responsable del 60 y 80 por ciento de los casos de demencia y pese a tener un origen desconocido tiende a debutar en personas mayores de 65 a�os, si bien, se han realizado estudios donde m�s de personas menores de esta edad han desarrollado Alzheimer en fases tempranas. Una de las principales caracter�sticas de esta enfermedad es su progresividad y consiguiente empeoramiento gradual.\\
Para poder catalogar la enfermedad del Alzheimer en distintas categor�as se emplean dos sistemas bien diferenciados. El primero contempla las tres fases principales de la enfermedad donde se distingue en: leve, moderado y grave. El segundo se realiza a trav�s de la escala de GDS (Escala de deterioro global) donde se consideran 7 etapas de las cuales hablaremos m�s adelante.\\
En fases tempranas los s�ntomas de esta enfermedad son la dificultad para recordar la informaci�n reci�n aprendida en casos m�s avanzados llega a producirse desorientaci�n, cambios en el comportamiento o la dificultad para realizar actividades basales como hablar, tragar o caminar.\\
Los cambios cerebrales comienzan mucho antes de que la enfermedad se presente como tal con las primeras p�rdidas de memoria, seg�n estudios se produce una p�rdida de acetilcolina que provoca da�os en las c�lulas cerebrales que a la larga se extiende produciendo que �stas pierdan la capacidad de trabajo por lo tanto mueran y se ocasionen da�os irreversibles.\\
Actualmente no hay cura para esta enfermedad, si bien, hay diversos tratamientos, en este Trabajo nos centraremos principalmente en los tratamientos no farmacol�gicos que ayudan a ralentizar el avance. Una de las principales terapias que se est�n implantando actualmente son las terapias de reminiscencia pero antes de ahondar en ello, debemos de conocer primero qu� es la reminiscencia, cu�les son sus objetivos y beneficios. 

\section{Fases de la enfermedad}

\section{Escala de deterioro global de Reisberg}

\section{Historias de Vida}
Se conoce como reminiscencia al recuerdo impreciso de un hecho o imagen del pasado que viene a la memoria. 
