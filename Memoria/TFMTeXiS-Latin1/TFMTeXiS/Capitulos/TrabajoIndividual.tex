% !TeX encoding = ISO-8859-1

\chapter{Trabajo Individual}
\label{cap:trabajoIndividual}

En este cap�tulo visualizaremos en que partes ha trabajo cada uno del equipo.

\section{Cristina Barquilla Blanco}
\begin{itemize}
	\item Redacci�n de los puntos Motivaci�n, Objetivos, Metodolog�a y Estructura de la Memoria de la Introducci�n.
	\item Redacci�n del TFG "Generaci�n de historias a partir de una base de conocimiento" de la secci�n de Trabajos relacionados.
	\item Transcripci�n de la primera parte de la Entrevista a uno de los expertos en terapias de reminiscencia.
	\item Captura de requisitos del proyecto en base a las entrevistas y transcripciones realizadas.
\end{itemize}
\section{Patricia D�ez Garc�a}
\begin{itemize}
	\item Redacci�n de los puntos Enfermedad del Alzheimer, Terapias basadas en reminiscencia.
	\item Redacci�n de Fases del Alzheimer, Escala de deterioro global de Reisberg, Mini-examen cognoscitivo (MEC) de Lobo e Historias de Vida.
	\item Transcripci�n de la primera parte de la Entrevista a uno de los expertos en terapias de reminiscencia.
	\item Captura de requisitos del proyecto en base a las entrevistas y transcripciones realizadas.
\end{itemize}
\section{Santiago Marco Mulas L�pez}
\begin{itemize}
\item {Redacci�n de la memoria}.
\end{itemize}
\section{Eva Verd� Rodr�guez}
\begin{itemize}
\item Preparaci�n inicial del documento Latex.
\item Redacci�n de la Introducci�n y parte de la motivaci�n
\item "Redacci�n del Sistema de asistencia para cuidado de enfermos del Alzheimer" en "Trabajos relacionados"
\item Transcripci�n de la primera parte de la Entrevista a uno de los expertos en terapias de reminiscencia.
\item Captura de requisitos del proyecto en base a las entrevistas y transcripciones realizadas.
\end{itemize}
