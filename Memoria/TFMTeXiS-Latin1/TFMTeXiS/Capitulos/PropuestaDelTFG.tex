% !TeX encoding = ISO-8859-1

\chapter{Recu�rdame}
\label{cap:propuestaDelTFG}

A lo largo de este cap�tulo redactaremos la captura de requisitos.

\section{Captura de Requisitos}

En este apartado se encuentran los distintos requisitos obtenidos de las entrevistas realizadas a los expertos.
\begin{itemize}
	\item Registrar un paciente y todos los datos de la entrevista inicial de cada etapa de su vida. Esto quiere decir que hay que registrar el GDS del paciente, si tiene deficiencia visual o de o�do y registrar las partes importantes de cada etapa de su historia de vida. Se tienen que poder registrar im�genes, v�deos y audios. A partir de todos estos recuerdos crear una historia de vida.
	\item Etiquetar las etapas de la historia de vida con Infancia, Adolescencia, Adulto joven, Adulto, Adulto mayor y categorizar los datos de la entrevista por hobbies, m�sica, familia, etc.
	\item Crear un buscador por paciente que permita filtrar por etapas y categor�as como hobbies, familia, etc.
	\item Mostrar la historia de vida completa de un paciente incluyendo im�genes, texto o audio registrada.
	\item Registrar las sesiones de terapia de un paciente indicando el recuerdo que se va a tratar en la sesi�n.
	\item Etiquetar ese recuerdo visto en la terapia con las etiquetas de recuerdo positivo, neutro o negativo y conservado, perdido o en riesgo de perder.
	\item Informe de seguimiento de la evoluci�n del paciente y registrar la evoluci�n mediante una escala del 0 al 10.
	\item Tener una versi�n de la aplicaci�n para los terapeutas y otra versi�n para los familiares del paciente. Para la versi�n de familiares: Poder visualizar un calendario con todas las actividades que pueden realizar los pacientes.
	\item Mensajer�a para comunicarse entre familiar y terapeuta.
	\item La aplicaci�n tiene que ser compatible para ordenadores y tablets.

\end{itemize}